#==============================================================================
# RCS Header:
#   File         [$Source: /home/hc0300/umui/srce_code/GHUI_archive/ghui2.0/doc/uiComponents.tex,v $]
#   Revision     [$Revision: 1.2 $]     Named [$Name: head#main $]
#   Last checkin [$Date: 2000/10/16 14:22:50 $]
#   Author       [$Author: hadsm $]
#==============================================================================

\section{Components Required by All GHUI-Based User Interfaces}

The directory structure of the UMUI, a typical GHUI-based user
interface is as follows. For version 2.0, the top-level directory is
called umui2.0. Within this directory the following directories exist:

\begin{description}
\item [Install]
Contains the Configure script that is run when installing the UMUI,
and a umui.def.template file which is used to create the umui.def
application definition file that is installed into the GHUI apps
directory. Among other things, the application definition file can
redefine some of the directory names in the version specific
directories.
\item [vn*]
There exists a separate directory for each version of the
application. The * usually stands for a version number (eg vn4.5,
vn5.1.1) but can stand for anything. Each version number needs to be
registered in the application definition file.
\item [updates]
Contains scripts that upgrade a job from one version to another.
\item [etc]
Created when the application is configured. Contains the client and
server definition files.
\item [bin]
Created when the application is configured. Contains the scripts for
starting the application and the application's Server Administration
user interface.
\item [help]
Intended to hold help on the upgrade processes but not implemented
fully yet.
\end{description}

Each version specific directory contains a number of standard
directories but can also contain additional application specific
directories. Each directory is registered in the application
definition file along with its symbolic name which the GHUI uses to
reference it. The symbolic name can be the same as the directory
name. The standard directories are as follows:

\begin{description}
\item [windows]
Holds the control files that define the input panels. Also holds the
nav.spec file that defines the hierarchy displayed by the Navigation
Tree, the nav.buttons file that defines the buttons to be displayed by
the Navigation Window.
\item [variables]
Holds the files that declare the application variables. May optionally
hold a partition.database file, a js_partition.database file for
applications that use the jobsheet function and a link_variables file
for applications that include variables that are functions of other
variables.
\item [functions]
Holds application-specific Tcl functions. Note that purely for
historical reasons the functions directory in the UMUI is called UM.

Application specific functions can be written to provide better
validation of user input, to write non-standard input panels, to
implement non-standard Navigation Window buttons among other
things. Applications can be written without any such functions though.
Knowledge of Tcl is required to write application specific functions.
\item [processing]
Holds template files that define how the job setup is converted to a
format appropriate for the related application. Files are written in a
simple formatting language.
\item [help]
Contains help for each of the input panels. Also used to hold help
files for the Navigation Window buttons.
\item [skeletons]
Holds templates for standard input panels.
\item [icons]
Holds bitmaps for icons.
\end{description}

NB some of the files in the icons, skeletons and help directories are
not always specific to a particular application and should probably
form part of the GHUI. This may happen at some stage. An idea that
would be suitable for the help files and the icons is that generic
help files are held in the GHUI but that they could be over-loaded by
help files with the same name in the application specific directories.

The following components are probably required by all GHUI-based user
interfaces.

\begin{itemize}
\item The GHUI needs to know some basic information about the user
interface when it starts up. Such information includes the location of
the user interface code, the list and type of information stored by
the server about each job and experiment, the versions of the user
interface available and what version of the Navigation System code
each version uses, and the list and type of information to be shown to
the user on the Entry System user interface. All this information is
registered in the application definition file in the apps directory of
the GHUI installation. An application definition file template should
be created in the Install directory to allow the application to be
configured.
\item Each user interface can have more than one version. Each version
is held in a separate subdirectory. Upgrading from one version to
another is allowed if an appropriate upgrade script exists in the
``updates'' directory at the top level of the user interface.
\item Each input panel is defined by an input panel control file. All
input panel control files are kept in the windows directory.
\item A skeleton file defines the general format of the input
panels. The normal format of the UMUI input panels, defined in
``entry.skel'', specifies a panel with three standard buttons: Abandon
(abandon changes and close panel), Close (check inputs, and if no
errors, save and close panel), Help (Display help panel if
available). A second format is the ``dummy.skel'' where no GHUI panel
appears. Instead the UI developer will define a procedure to be
executed when the panel is selected. In the UMUI, dummy panels are
used to call up the non-GHUI STASH user interface.
\item Ideally, each panel has its own help file in the help
directory. A help file has a .help suffix and should be prefixed with
the the window identity (defined by the .winid line in the panel
control file and conventionally the same as the prefix of the panel
control file itself).
\item The nav.spec file in the windows directory defines the panel
hierarchy.
\item The nav.buttons file in the windows directory defines the list
of function buttons displayed in the Navigation Window.
\item The variables directory holds three variable registers:
parameter.register, system.register and var.register; and two variable
databases: the parameter.database and the system.database which define
the values of the variables declared in the first two registers. The
variables in the var.register are those which define job setups; each
separate job setup is stored in a file known as a job basis file in
the central database.
\item The GHUI incorporates a processing function. Template files in
the processing directory define a text format for output files
appropriate for use with the related model. Use of logic and functions
allows the output format to be quite flexible. More than one output
file can be defined. The GHUI processing file requires a top-level
file called ``top''. This file can refer to other processing control
files.
\end{itemize}
