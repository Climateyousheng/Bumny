\section{Introduction}

This document contains the system requirements and installation
instructions for the Generic Hierarchical User Interface (GHUI) and
the GHUI-based user interfaces. The GHUI is the basis code required by
user interfaces such as the Unified Model User Interface (UMUI).  The
GHUI should work on any Unix system supporting X windows (X11R5 or
later).

The GHUI is written using the Public Domain Tcl/Tk language and a
small amount of C code that requires compiling.

\section{System Requirements}
\begin{itemize}
\item UNIX operating system with standard Korn shell;
\item ANSI complient C compiler;
\item X windows X11R5 or later;
\item Sufficient disk space on each client to store the GHUI code and
the code relating to the user interface being installed. Furthermore,
machines that host servers need space to store the application
database; the central repository of jobs.
\end{itemize}

\section{System Overview}
Much of the system documentation is available as help files to the
various user interfaces within the GHUI system. This section aims to
give a deeper explanation of the procedures without going into
specific details that may evolve from application to application.

The GHUI consists of three main parts. These are as follows:
\begin{itemize}
\item Job database server system. The system can be run with one
server or with primary and backup server. There is generally one set
of servers and one database (or identical primary and backup database)
per user interface.

The job servers keep track of all experiments and jobs in the
system. All access to job setups is through the network interface of
these servers. The machines that host servers should have enough disk
space to hold all job files. It should be accessible over the network
to all machines that will be running the client software. It should be
able to run the server process continuously (this can make a queue
based time sharing supercomputer unsuitable). The job server is not
CPU or I/O intensive, so a basic workstation is more that adequate.

The server system has an Administration user interface that is used to
start, stop and configure the clients and servers.

\item Entry system. The Entry System is the user interface that
appears when the application is first started. It displays lists of
experiments and jobs held the database. It is configurable by the
system manager to display some or all of the basic information stored
by the server about each job. There is also a Filter system that
allows users to display subsets of experiments and jobs. Generally
when it is first started it displays the experiments owned by the
user.

\item Navigation System. This is a forms-based user interface system
for editing a job. It is called when a job is opened. Its name comes
from the fact that its main component is a hierarchical tree that
groups input panels into categories and subcategories and thus allows
easy navigation from one part of the interface to another. It is
sometimes referred to as the Job Editor.

\end{itemize}

A GHUI-based user interface is a network based client-server
application. The job server is the ``server'' in this system and the
entry system, Navigation System and Administration user interface are
``clients''. Only one server (or one primary and backup server) should
run on one machine (or two separate machines) on the network. Many
clients like can be run from anywhere on the network including the
machine running the server.

\section{Installation Procedure}

Ideally, the software components required to install the user
interface would have included a README file to explain how to install
the application. The following may duplicate much of this information
but generally the README file will contain more specific and up to
date information.

A GHUI-based user interface requires three software components. The
components should be installed in the order given:

\begin{enumerate}
\item An appropriate version of Tcl/Tk.
\item The GHUI software.
\item The user interface files for the GHUI-based application
\end{enumerate}

The components are usually provided as separate tar files that need to
be untarred before installing. The Tcl/Tk and GHUI packages generally
need to be configured, compiled and then installed. The GHUI-based
application usually requires configuring only.

For example, to install the GHUI run the ``Configure'' script,
probably in the ``Install'' directory of the untarred directory
structure. Answer the series of questions relating to the locations of
the Tcl/Tk installation, the compilation options and where you want
the GHUI code to be installed. Once you are happy with the settings
the Configure script creates a set of Makefiles for compiling and
installing the code and also some template files that are required
when configuring a GHUI-based application. The Configure script also
saves the responses in a file called config.cache. If the Configure
script is run again these responses will be offered as the default
answers.

Run ``make'' to compile the few bits of C code and ``make install'' to
copy all the files to the chosen location. Once installed, the
untarred GHUI files can be discarded, though keeping the config.cache
file allows the process to be repeated more quickly if it needs to be.

The GHUI-based application requires configuring only; there are no
further compilation or installation steps so untar it to a the
location where it is to be kept. Information required to configure a
GHUI-based application includes the location of the GHUI installation
and information about the chosen server configuration. The user
installing a GHUI-based application needs to have write access to the
apps directory of the GHUI installation to which a small application
definition file is written.

The last question in the Configure scripts for the GHUI and the
GHUI-based application is to ask for confirmation that the inputs are
correct. Confirmation is required before anything else happens.  If
confirmation is not given the script will still create the
config.cache file which will be used as the default answer in
subsequent runs of the script.

Configuring the GHUI-based application creates a number of files whose
names are related to the application name provided as part of the
configuration. For the UMUI, the following files would be created:

\begin{itemize}
\item A script to start the application called ``umui'' held in a
directory called bin in the top level UMUI directory.
\item A script to start the Server Administration user interface
called ``umui_admin'' also held in the bin directory.
\item A ``servers.def'' file that defines the locations of the
servers. This file is held in the etc in the top level UMUI directory.
\item A ``clients.def'' file that lists all the locations of all UMUI
clients. Also held in the etc directory.
\item An application definition file called ``umui.def'' that is
copied to the GHUI apps directory. This file is based on a
umui.def.template that should have been provided as part of the UMUI
installation package. The configuration just adds the location of the
GHUI and UMUI code.

Essentially the umui script starts up the GHUI providing it with only
one piece of application specific information; the application name
``umui''. The GHUI then finds all the information it needs to run the
UMUI from the umui.def file.
\end{itemize}

\subsection{Installing on More Than One Machine}

When installing on multiple clients set up .rhosts files for the
userid on each client to allow it to be accessed by other
clients. Note that when using primary and backup servers, both servers
should run under the same account name.

After the first installation a tip for speeding up subsequent
installations is to copy the config.cache file from the first
installation to the Install directory on subsequent clients; hopefully
the answers for each installation will mostly be the same. 

All installations of the GHUI-based application need to know the
location of the servers. The server information is kept in a file
called ``servers.def'' in a directory called ``etc''. Configuration of
the second and subsequent installations is simplified as the Configure
script for the application will request a location of a previous
installation and simply copy the servers.def file from that location
(assuming it has permission to rcp it from the remote host).

Additionally, each client ought to know the locations of all other
clients. Client information is required by the Server Administration
user interface when the server definition is changed as a new version
of the servers.def file then needs to be copied to all other clients.
The client list is maintained in a file called ``clients.def'' also
held in the etc directory. Again, when making multiple installations
the second and subsequent configuration will copy this information
from the client name supplied to it. It will then add the name of the
new client to the list and distribute the new version of the file to
all the other client installations.

In short, the clients.def file and the servers.def file should always
be identical on all client installations. See the help in the
Administration user interface for examples of when and how the server
information might be changed.

If installing to the same pathname on multiple clients then all
installations will be identical. That is, it is reasonably safe to
install on one client and then copy the installations to the other
clients. However, do update all the clients.def files with a list of
all clients.

\end{document}
