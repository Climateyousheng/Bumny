#==============================================================================
# RCS Header:
#   File         [$Source: /home/hc0300/umui/srce_code/GHUI_archive/ghui2.0/doc/GHUI_doc.tex,v $]
#   Revision     [$Revision: 1.2 $]     Named [$Name: head#main $]
#   Last checkin [$Date: 2000/10/16 14:24:06 $]
#   Author       [$Author: hadsm $]
#==============================================================================

\documentclass{report}
\usepackage[dvips]{graphicx}
\begin{document}
\begin{titlepage}
\begin{center}
\textbf{The General Hierarchical User Interface}
\ \\
S.D.Mullerworth \\
\ \\
Version 1.0 \\
\ \\
20th October 2000 \\
\ \\
GHUI version 2.0. Navigation System version 1.2

\vspace{2.5in}
Climate Research \\
The Met. Office \\
London Road \\
BRACKNELL \\
Berkshire \\
RG12 2SY \\
United Kingdom

\vspace{0.5in}
(c) Crown Copyright 2000
\par
\ \\
This document has not been published. Permission to quote from
it must be obtained from the Head of Climate Prediction at the above address.
\end {center}

\end{titlepage}

\tableofcontents
\chapter{Modification History}

  \begin{center}
  \begin{tabular}{|c|c|c|p{0.25\hsize}|}           \hline
  \textbf{Date} & \textbf{Author} & \textbf{UM version}
  & \textbf{Description of change} \\ \hline

    20 October 2000 &  S.D.Mullerworth & 5.0 &
    Original. \\ \hline

  \end{tabular}
  \end{center}

\chapter{Introduction}

The Generic Hierarchical User Interface, or GHUI, is a package that
allows the creation of large forms based user interfaces. It is the
package that underlies user interfaces developed at the Met. Office
such as the UMUI, VARUI and SCSUI.

The GHUI allows the user interface developer to write input panels
using a simple language. Panels are presented to users in a tree
hierarchy to help them find their way to the appropriate panel or
setup panel. Validation of user input is performed to reduce the
number of user errors. Once setup is completed by the user a
processing function converts the input into a suitable output as
prescribed by the user interface developer.

Users may own many separate job setups. The GHUI incorporates code for
setting up a central job database in which all jobs owned by all users
is stored. Jobs are accessed via a client/server system.

The functionality provided by the GHUI aims to be reasonably
comprehensive but also aims to be extendible; where the GHUI functions
do not provide a comletely appropriate validation check or panel
design an alternative one can be provided. Some knowledge of Tcl and
Tk may be required to implement such extensions.

This document incorporates both system documentation, documentation
required to develop new user interfaces and documentation required to
install and manage user interfaces. Firstly, an overview of the
look-and-feel of a typical GHUI based user interface is given.

#==============================================================================
# RCS Header:
#   File         [$Source: /home/hc0300/umui/srce_code/GHUI_archive/ghui2.0/doc/overviewDoc.tex,v $]
#   Revision     [$Revision: 1.2 $]     Named [$Name: head#main $]
#   Last checkin [$Date: 2000/10/16 13:18:25 $]
#   Author       [$Author: hadsm $]
#==============================================================================

\chapter{Overview of the GHUI}

The GHUI was designed to solve the problem of setting up runs of the
Unified Model (UM) at the Met. Office. The UM is a suite of
atmospheric and oceanic numerical modelling software developed at the
Met. Office. It is the major tool for weather forecasting, climate
prediction and research. The UM supports models on global and regional
domains over a wide range of temporal and spatial scales. It runs on a
range of platforms including vector parallel, massively parallel and
UNIX workstations. All aspects of a model run are handled by the UM
system including code compilation, data archiving and management of
very long climate simulations.

The comprehensive and flexible nature of the UM system means that
several hundred input parameters are required to specify a model run,
and different types of run may require significantly different sets of
input. Additionally, the form of the input to a UM run is not
straightforward; a UNIX script and a set of files containing various
scientific and control parameters are required. Therefore, a suitable
user interface needed to fulfil the following requirements:

\begin{itemize}
\item To contain a large number of input windows each of which has a
number of questions.
\item To group together related windows to ensure that users can find
and access a relevant window or a set of relevant windows with ease.
\item To allow users to copy and save job specifications. Thus, by
basing a job on a similar job copied from elsewhere, the amount of
additional input required is kept reasonably small.
\item To incorporate functionality for processing users' input into
the appropriate format for controlling a UM run.
\end{itemize}

Furthermore a user interface also needed to find the appropriate line
between being simple enough to enable easy development and maintenance
of the system and being complex enough to satisfy the requirements of
the UM. The result of the project was the development of the Unified
Model User Interface (UMUI). The code behind the UMUI was developed
with reusability very much in mind and after a while the reusable, or
generic, components were extracted to create the a package of
applications and functions called the Generic Hierarchical User
Interface.

The philosophy behind the design of the GHUI was to include within the
package all the functionality needed to provide most of the user
interface functionality. But it was also designed to be highly
configurable and easily extendible so that further functionality could
be added where the standard GHUI functions were too limited or
inappropriate.

A common use of application-specific code in the UMUI is to provide
alternative validation checks of user input. The GHUI provides a
number of primitive checking algorithms, for example to ensure that
input is of the correct data type, that numerical input is in the
correct range, that string input is within a given length and so on.
More complex checks can however be provided by supplying an
application-specific function. The simplest case of such a check is
where the input of one value is dependent on the input of another,
perhaps in a different panel. The cross-checking routine will be able
to provide the error message and perhaps also the location of the
dependent question.

Even application-specific input panels can be created.  The GHUI
provides functionality to develop reasonably basic forms-based panels,
and use of simple layout tools and the ability to deactivate or hide
non-relevant questions mean that they are easy to read and acceptable
for most purposes. However, in the UMUI one particularly complex set
of input required its own separate user interface. This
application-specific user interface was ``hooked'' into the UMUI and
appears in much the same way as a normal panel when the appropriate
item in the main Navigation window is selected.

The GHUI package comprises the following main components:

\begin{itemize}
\item A client/server system that allows all users to store each of
their separate setups (termed ``jobs'') in a single central database
and allows users to view anyone else's jobs. The system groups jobs
together into sets called ``experiments''.

Each experiment is referred to with a 4-letter identity and each job
is referred to by a single letter. Thus a job is fully described by 5
letters, for example as ababc or sometimes abab#c to reference job
``c'' in experiment ``abab''. At the time of writing the structure of
the database is very simple. Within the database directory, each
experiment comprises a small text file (\eg abab.exp) that summarises
the contents of the experiment and a directory (\eg abab) that holds
the information about each job. Within the experiment directory there
exists two files per job; a small text file that summarises the
contents of the job (eg c.job) and a ``basis database'' with a one
letter filename which contains the full information about the job
setup. Thus the database format is very simple and all files within
are text files with reasonably straightforward formats. (If gzip
compression is available on the server machine, job basis databases
will be compressed. So the file name will have a .gz extension).

When the server process is started it reads and stores the basic
information about each experiment and job held in the .exp and .job
files.  The clients can query this information. Some of the
information, such as version number or owner, is generic. Other
information is application-specific; it is defined by the particular
user interface configuration and depends on the setup of the
particular job.

A Server Administration user interface manages the server system. The
system can be run with one server or two. With two servers
(dual-server mode) the secondary server acts as a backup for when the
primary server fails. Only the user interface manager can use this
interface.

\item The Entry System, so called because it is the interface that
appears when the user interface is first started. It displays a
selection of experiments and jobs depending on the current search
settings. Thus it enables only a certain subset of all experiments to
be displayed. By default, when the Entry System is started it will
display all experiments owned by the current user. The Entry System
can be configured by the system administrator to display some or all
of the basic job and experiment information stored by the server.

The Entry System provides functionality to copy, delete or create jobs
and experiments, to upgrade jobs to the next version of the user
interface and to compare two jobs.

\item The Navigation System is sometimes called the job edit
system. It allows the setup of a job to be changed. The Navigation
System is so called because it is designed to allow the user to
navigate easily from one part of the system to another. By using a
hierarchical tree structure, the Navigation Tree, to categorise panels
it can sensibly manage access to hundreds of input panels. For example
the UMUI has over 200 panels.

The main Navigation System window displays the Navigation Tree on its
left-hand of two main panels. Each node of the tree describes a
category or subcategory of panels and/or nodes. When a node is
selected, the panels relating to that node are listed on the
right-hand panel of the window. When an item from the list is
selected, the appropriate panel is generated and displayed.

The Navigation System provides access to a range of functions through
an array of buttons displayed along the bottom of the main window. The
main functions include buttons for saving work and for quitting the
session. Other important functions include the processing function
that converts the job setup into a format suitable for the related
application and the Check Setup function that runs a check on the
current setup and reports any error. The array of buttons that are
displayed is configurable by the user interface developer.

\end{itemize}

\subsection{Summary of the Components of the GHUI}

The directory structure of the GHUI is as follows. For version 2.0,
the top-level directory is called ghui2.0. Within this directory the
following directories exist:

\begin{description}
\item [Install]
Contains the Configure script required to configure the GHUI
installation. Also contains template files that are used to generate
the application startup scripts (eg the ``umui'' and ``umui_admin''
scripts for the UMUI).
\item [apps]
Contains an application definition file for each user interface
registered to use the GHUI.
\item [bin]
Contains some general GHUI scripts such as those required to start the
server. Some of these scripts are generated from template scripts when
the GHUI is configured. The template scripts are in the same directory
and have the same name but with a ``.in'' extension.
\item [tcl]
Contains the Tcl/Tk code for running the Entry System and Server
Administration interfaces, and Client/Server system.
\item [pkg]
Contains some more general Tcl packages also required for running the
Entry System and Server Administration interfaces, and Client/Server
system.
\item [help]
Contains the help files accessible from the Entry System and Server
Administration interfaces.
\item [icons]
Contains bitmap files that describe the icons used in the Entry System.
\item [doc]
Holds the documentation files.
\item [vn1.1, vn1.2 etc]
There can be more than one version of the Navigation System code. Each
version of the code has its own subdirectory.
\end{description}

Each directory holding a version of the Navigation System code
contains the following subdirectories:

\begin{description}
\item [tcl]
Contains the Tcl/Tk code for running the Navigation System.
\item [pkg]
Contains other Tcl packages and shared libraries for running the
Navigation System.
\item [help]
Contains help files accessible from the main Navigation System window.
\item [src]
Contains the C code for the shared libraries. This directory is
available within the installation package but is not copied to the
installed GHUI.
\item [doc] Holds version specific documentation (none available for
vn1.1 though).
\end{description}

NB at some point the pkg directories for the Entry System and for the
versions of the Navigation System may be rationalised to one
directory.

\input{sysManageDoc.tex}
\input{userDoc.tex}
\input{systemDoc.tex}
\end{document}
